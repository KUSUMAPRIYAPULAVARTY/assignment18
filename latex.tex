\documentclass[journal,12pt,twocolumn]{IEEEtran}

\usepackage{setspace}
\usepackage{gensymb}

\singlespacing


\usepackage[cmex10]{amsmath}

\usepackage{amsthm}

\usepackage{mathrsfs}
\usepackage{txfonts}
\usepackage{stfloats}
\usepackage{bm}
\usepackage{cite}
\usepackage{cases}
\usepackage{subfig}

\usepackage{longtable}
\usepackage{multirow}

\usepackage{enumitem}
\usepackage{mathtools}
\usepackage{steinmetz}
\usepackage{tikz}
\usepackage{circuitikz}
\usepackage{verbatim}
\usepackage{tfrupee}
\usepackage[breaklinks=true]{hyperref}
\usepackage{graphicx}
\usepackage{tkz-euclide}

\usetikzlibrary{calc,math}
\usepackage{listings}
    \usepackage{color}                                            %%
    \usepackage{array}                                            %%
    \usepackage{longtable}                                        %%
    \usepackage{calc}                                             %%
    \usepackage{multirow}                                         %%
    \usepackage{hhline}                                           %%
    \usepackage{ifthen}                                           %%
    \usepackage{lscape}     
\usepackage{multicol}
\usepackage{chngcntr}

\DeclareMathOperator*{\Res}{Res}

\renewcommand\thesection{\arabic{section}}
\renewcommand\thesubsection{\thesection.\arabic{subsection}}
\renewcommand\thesubsubsection{\thesubsection.\arabic{subsubsection}}

\renewcommand\thesectiondis{\arabic{section}}
\renewcommand\thesubsectiondis{\thesectiondis.\arabic{subsection}}
\renewcommand\thesubsubsectiondis{\thesubsectiondis.\arabic{subsubsection}}


\hyphenation{op-tical net-works semi-conduc-tor}
\def\inputGnumericTable{}                                 %%

\lstset{
%language=C,
frame=single, 
breaklines=true,
columns=fullflexible
}
\begin{document}


\newtheorem{theorem}{Theorem}[section]
\newtheorem{problem}{Problem}
\newtheorem{proposition}{Proposition}[section]
\newtheorem{lemma}{Lemma}[section]
\newtheorem{corollary}[theorem]{Corollary}
\newtheorem{example}{Example}[section]
\newtheorem{definition}[problem]{Definition}

\newcommand{\BEQA}{\begin{eqnarray}}
\newcommand{\EEQA}{\end{eqnarray}}
\newcommand{\define}{\stackrel{\triangle}{=}}
\bibliographystyle{IEEEtran}

\providecommand{\mbf}{\mathbf}
\providecommand{\pr}[1]{\ensuremath{\Pr\left(#1\right)}}
\providecommand{\qfunc}[1]{\ensuremath{Q\left(#1\right)}}
\providecommand{\sbrak}[1]{\ensuremath{{}\left[#1\right]}}
\providecommand{\lsbrak}[1]{\ensuremath{{}\left[#1\right.}}
\providecommand{\rsbrak}[1]{\ensuremath{{}\left.#1\right]}}
\providecommand{\brak}[1]{\ensuremath{\left(#1\right)}}
\providecommand{\lbrak}[1]{\ensuremath{\left(#1\right.}}
\providecommand{\rbrak}[1]{\ensuremath{\left.#1\right)}}
\providecommand{\cbrak}[1]{\ensuremath{\left\{#1\right\}}}
\providecommand{\lcbrak}[1]{\ensuremath{\left\{#1\right.}}
\providecommand{\rcbrak}[1]{\ensuremath{\left.#1\right\}}}
\theoremstyle{remark}
\newtheorem{rem}{Remark}
\newcommand{\sgn}{\mathop{\mathrm{sgn}}}
\providecommand{\abs}[1]{\left\vert#1\right\vert}
\providecommand{\res}[1]{\Res\displaylimits_{#1}} 
\providecommand{\norm}[1]{\left\lVert#1\right\rVert}
%\providecommand{\norm}[1]{\lVert#1\rVert}
\providecommand{\mtx}[1]{\mathbf{#1}}
\providecommand{\mean}[1]{E\left[ #1 \right]}
\providecommand{\fourier}{\overset{\mathcal{F}}{ \rightleftharpoons}}
%\providecommand{\hilbert}{\overset{\mathcal{H}}{ \rightleftharpoons}}
\providecommand{\system}{\overset{\mathcal{H}}{ \longleftrightarrow}}
	%\newcommand{\solution}[2]{\textbf{Solution:}{#1}}
\newcommand{\solution}{\noindent \textbf{Solution: }}
\newcommand{\cosec}{\,\text{cosec}\,}
\providecommand{\dec}[2]{\ensuremath{\overset{#1}{\underset{#2}{\gtrless}}}}
\newcommand{\myvec}[1]{\ensuremath{\begin{pmatrix}#1\end{pmatrix}}}
\newcommand{\mydet}[1]{\ensuremath{\begin{vmatrix}#1\end{vmatrix}}}

\numberwithin{equation}{subsection}

\makeatletter
\@addtoreset{figure}{problem}
\makeatother
\let\StandardTheFigure\thefigure
\let\vec\mathbf

\renewcommand{\thefigure}{\theproblem}

\def\putbox#1#2#3{\makebox[0in][l]{\makebox[#1][l]{}\raisebox{\baselineskip}[0in][0in]{\raisebox{#2}[0in][0in]{#3}}}}
     \def\rightbox#1{\makebox[0in][r]{#1}}
     \def\centbox#1{\makebox[0in]{#1}}
     \def\topbox#1{\raisebox{-\baselineskip}[0in][0in]{#1}}
     \def\midbox#1{\raisebox{-0.5\baselineskip}[0in][0in]{#1}}
\vspace{3cm}
\title{Assignment 18}
\author{KUSUMA PRIYA\\EE20MTECH11007}

\maketitle
\newpage

\bigskip
\renewcommand{\thefigure}{\theenumi}
\renewcommand{\thetable}{\theenumi}
Download codes from 
%
\begin{lstlisting}
https://github.com/KUSUMAPRIYAPULAVARTY/assignment18
\end{lstlisting}
%
 
\section{QUESTION}
Use theorem 20 to prove the following.If $\vec{W}$ is a subspace of a finite dimensional vector space $\vec{V}$ and if $\cbrak{g_1,g_2,\hdots,g_r}$ is any basis for $\vec{W}^0$ then
\begin{align}
    \vec{W}=\bigcap_{i=1}^r \vec{N}_{g_i}
\end{align}

%

\section{Theorem 20}
Let $g,f_1,f_2,\hdots,f_r$ be linear functionals on vector space $\vec{V}$ with respective null spaces $\vec{N},\vec{N}_1,\hdots,\vec{N}_r$.\\Then $g$ is a linear combination of $f_1,\hdots ,f_r$ if and only if $\vec{N}$ contains the intersection $\vec{N}_1 \cap \vec{N}_2\cap \hdots \cap \vec{N}_r$.
\section{Solution}
\begin{table}[!h]
\centering
\begin{tabular}{|p{3cm}|p{5cm}|}
\hline
\textbf{PARAMETERS}&\textbf{DESCRIPTION}\\
\hline
$\vec{V}$& Finite dimensional vector space with $dim(\vec{V})=n$\\
\hline
$\vec{W}$ & Subspace of $\vec{V}$\\
\hline
$\vec{W}^0$ & Annihilator of $\vec{W}$\\
\hline
$g \in \vec{W}^0$ & $g(\vec{w})=0 \forall \vec{w} \in \vec{W}$\\
\hline
$\cbrak{g_1,g_2,\hdots,g_r}$& basis for $\vec{W}^0$\\
\hline
$\vec{N}_{g_i}$ & Null space of $g_i$\\
\hline
\end{tabular}
\caption{Input Parameters}
\end{table}
Any $g \in \vec{W}^0$ can be written as
\begin{align}
    g=\sum_{i=1}^r c_ig_i
\end{align}
So, from the theorem, if $\vec{N}$ is nullspace of $g$
\begin{align}
    \bigcap_{i=1}^r \vec{N}_{g_i} \subseteq \vec{N}\label{1}\\
    \forall \vec{w} \in \vec{W},g(\vec{w})=0\label{2}\\
   \eqref{1},\eqref{2} \implies \vec{W} \subseteq \bigcap_{i=1}^r \vec{N}_{g_i}\label{3}
\end{align}
To prove using method of contradiction, assume
\begin{align}
   \vec{W} \neq \bigcap_{i=1}^r \vec{N}_{g_i} \label{4}
\end{align}
From \eqref{3},\eqref{4} there exists a vector  
\begin{align}
 \vec{e} \in  \bigcap_{i=1}^r \vec{N}_{g_i}, \vec{e} \notin \vec{W}\\
 \text{So  }\forall g \in \vec{W}^0, g(\vec{e})=0 \label{5}
\end{align}
Since $g$ is a linear functional on $\vec{V}$,
\begin{align}
    \eqref{5} \implies \vec{e} \in \vec{V}
\end{align}
Let us define a functional $f$ on $\vec{V}$ such that 
\begin{align}
    f(\alpha)=
    \begin{cases}
         0 & \text{ for }     \alpha \in \vec{W}\\
        \neq 0 & \text{ for } \alpha \in \vec{V},\alpha \notin \vec{W}
    \end{cases}\\
  \therefore f \in \vec{W}^0 \text{ and } f(e) \neq 0  
\end{align}
which contradicts with \eqref{5}\\
Hence 
\begin{align}
     \vec{W}=\bigcap_{i=1}^r \vec{N}_{g_i}
\end{align}
\end{document}


